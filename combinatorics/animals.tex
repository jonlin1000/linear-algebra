In this section we will highlight some results made in the combinatorial study of \textit{lattice animals}.

\begin{definition}
A polyomino is a non-empty union of lattice squares with connected interior.
\end{definition}

\begin{theorem}
Let $a_n$ be the number of polyominos of area $n$. Then $a_n^{1/n}$ convergest to some limit $\mu$ (called Klarner's constant). 
\end{theorem}
 By numerical experiements it seems like $\mu \approx 4.06$. The best known bounds for $\mu$ are $4.002 < \mu < 4.65$.

From a polyomino we can associate it with a lattice animal: (image here).

\begin{definition}
A directed animal is an animal in which there exists a special node called the source such that every other point can be reached from the source using a NE path (a path which goes only north or east).
\end{definition}

We know better the asymptotics for directed animals, by this theorem of Dhar.

\begin{theorem}
Let $b_n$ be the number of directed animals with $n$ nodes. Then $b_n^{1/n} \to 3$. 
\end{theorem}

This was proven by M\'elou and Rechnitzer. The idea is to associate these directed animals with ``dimer stacks'' and let them fall by gravity. These dimers form a heap.

\begin{definition}
A pyramid is a non-empty heap of dimers with a unique minimal element. A half-pyramid is a pyramid in which no dimer lies to the right of the dimer on the bottom.
\end{definition}
\begin{theorem}
This (where this will be explained eventually) mapping gives a bijection between directed animals and pyramids of dimers.
\end{theorem}

\begin{theorem}
Let 
\[P(x) = x + 2x^2 + 5x^3 + 13x^4 + 35x^5 + \cdots\] be the generating function for nonempty pyramids of $n$ dimers, and 
\[Q(x) = x + x^2 + 2x^3 + 4x^4 + 9x^5\]
be the generating function for half pyramids. Then
\[P(x) = Q(x) + P(x)Q(x)\] and
\[Q(x) = x + xQ(x) + xQ(x)^2.\]

Then we can solve and we find that 
\[Q(x) = \frac{1 - x - \sqrt{(1 + x)(1 - 3x)}}{2x},\]
\[P(x) = \frac{\sqrt{\frac{1 + x}{1 - 3x}} - 1}{2}.\]
\end{theorem}
\begin{proof}
For the first equation, we observe that any pyramid is either a half pyramid, or a pyramid which is sitting on top of a half pyramid. For the second equation we decompose the half pyramids. Either we have one dimer, we have one dimer and sitting on top of it is a half pyramid, or sitting on top of it is a pyramid. The third row right entry must be a half pyramid (or else the pyramid would not be a half pyramid). These give the generating functional relations as desired.
\end{proof}

Finally, we use the following theorem from complex analysis:
\begin{theorem}
If $a_0 + a_1x + a_2x^2 + \cdots$ is the Taylor expansion about $z = 0$ of $f(z)$ that is analytic with a pole at $|z| = r$, then the series has radius of convergence $r$ and $a_n^{1/n} \to 1/r$ as $n \to \infty$.
\end{theorem}

We note that $P(x)$ has a branch point at $x = -1$ and a pole at $x = 1/3$. Dhar's Theorem immediately follows, as desired.