The sequence $\{F_n\}$ of fibonacci numbers is annihilated by the operator
\[T^2 - T - I\]
where $T$ is the left shift operator. The generating function has denominator $1-x-x^2$. Note that the coefficients appear flipped. This is true in genreal, as the following theorem indicates.
\begin{theorem}
Suppose $a_0, \dots, a_d$ are non-zero coefficients, and that $f$ is some sequence. The following statements are equivalent:
\begin{itemize}
	\item $f$ satisfies the equation
	\[(a_dT^d + a_{d-1}T^{d-1} + \cdots + a_1T + a_0)f = 0.\]
	\item The generating function for $f$ can be expressed as a rational function whose numerator is of degree less than $d$, with denominator
	\[a_d + a_{d-1}x + \cdots + a_0x^d\]
	\item $f(n)$ is expressible as a linear combination of functions of the form $n^jr^n$ where $r$ is a root fo $a_dt^d + \cdots + a_1t + a_0 = 0$ and where $j \geq 0$ is less than the multiplicity of $r$.
\end{itemize}
\end{theorem}

Observe the following: if the roots of such a polynomial are $\pm 1$, then the solutions to such a recurrence relation are simply polynomial equations in the variable $n$.

Let's consider some of these results in action. Suppose we'd like to find either a recurrence or closed form formula for some sequence, say,
\[1, 2, 5, 12, 29, 70, \cdots\] satisfying a linear recurrence relation.

If we suspect that such a relation is second order, we can take the system of equations

\begin{align*}
	5 &= A(2) + B(1) \\
	12 &= A(5) + B(2)
\end{align*}

From which we get $A = 2$ and $B = 1$. This gives us the candidate linear recurrence relation $f(n) = 2f(n-1) + f(n-2)$, which can be seen to work. But what if we suspected that the relation was third order instead? Then we get the linear recurrence 
\[f(n) = 3f(n-1) - f(n-2) - f(n-3).\]
Is there any relation between these two recurrence relations? It seems like the former is ``simpler'' because it involves less previous terms. It turns out that this intuition is correct. Here are some facts about the kinds of sequences you get.
\begin{proposition}
\begin{enumerate}
	\item Suppose $p$ and $q$ are polynomials such that $p(T)f = 0$ and $p \mid q$. Then $q(T)f = 0$.
	\item For any sequence $f$ satisfying an LRE with constant coefficients, there exists a unique monic polynomial $p(t)$ such that $p(T)f = 0$ and $q(T)f = 0$ implies that $p \mid q$. 
	\item Any sequence satisfying an LRE satisfies a linear recurrence relation satisfies a recurrence relation with a minimal number of terms and which is unique up to scalar multiples.
\end{enumerate}
\end{proposition}
\begin{proof}
The first item is more or less obvious. The second item follows from the fact that the set of polynomials $p$ such that $p(T)f = 0$ form an ideal in the polynomial ring and there is a unique monic polynomial which is the generator for said ideal. The third item more or less follows from the second item.
\end{proof}

For example, in the above example, the polynomial operators for each recurrence relation are
\begin{align*}
T^2 - 2T - I & \\
T^3 - 3T^2 + T + I &= (T^2 - 2T - I)(T - I)
\end{align*}

Let's consider a less trivial example. Let's find a recurrence relation for the sequence $g(n) = F_n + 2^n$, where $F_n$ is the $n$th Fibonacci number. If we suspect that this sequence satisfies a third order linear recurrence relation, we can solve a 3 dimensional system to get that 
\[g(n) = 3g(n-1) - g(n-2) - 2g(n-3).\] One can confirm that $g(n)$ indeed satisfies the recurrence relation above.

Here is a slicker method: We might use linearity, and observe that $(T^2 - T - I)(T - 2I)$ annihilates the $g(n)$. This gives us the desired recurrence relation. Or similarly, we add
\[\frac{1}{1 - x - x^2} + \frac{1}{2x} = \frac{2 - 3x - x^2}{1 - 3x + x^2 + 2x^3}.\]
To test the minimality of this recurrence relation, we can show that $\gcd(2 - 3x - x^2, 1 - 3x + x^2 + 2x^3) = 1$, or we can show that the determinant of cycles of the first equation is non-zero (that there are no non-trivial dependence relations in this recurrence, showing it is minimal).

Finally, we conclude this section with some combinatorics: we will tile a $2 \times n$ rectangle by dominos (known as a domino tiling).

This is a classic problem. We can solve it with the following observation. Either the $2 \times n$ rectangle begins with a vertical rectangle or two horizontal rectangles. Thus for $n \geq 2$ the number of rectangles $D_n$ satisifies the recurrence relation
\[D_n = D_{n-1} + D_{n-2}.\] It follows that the number of tilings is counted by the Fibonacci numbers.

What about a $3\times n$ rectangle? (to be illustrated later).