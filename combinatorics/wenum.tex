Previously, we looked at the number of ways to tile a $2 \times n$ rectangle with dominoes, and we found that the number of such tilings was a Fibonacci number. Here is another question: If we tile a $2 \times n$ rectangle with dominoes, then what fraction of the tiles will be vertical? This question is not exactly formed, as it depends on the tiling. Here is a better question: What is the expected number of tilings? Here's the magic technique that we will use: we will count with polynomials.

\begin{definition}
Given any $2 \times n$ tiling, give it weight $w^k$ if it has exactly $k$ vertical dominoes. The \textbf{number} of tilings by dominoes is defined as the sum of the weights. We denote this polynomial by $p_n(w)$. 
\end{definition}

For example, for a $2 \times 3$ tiling the number of tilings is $w^3 + 2w$. If we evaluate using $w = 1$ we get the actual \textit{number} of tilings. Here are the first few examples of tilings.

\begin{align*}
p_0(w) &= 1 (?) \\
p_1(w) &= w \\
p_2(w) &= w^2 + 1 \\
p_3(2) &= w^3 + 2w.
\end{align*}

Can we find a recurrence for $p_n(w)$ similarly to how we found a recurrence for ordinary tilings of the $2 \times n$ rectangle? It turns out that we can. Here is the reasoning: We can decompose any $2 \times n$ tiling as a tiling where there is one vertical tile on the left or $2$ horizontal tiles on the right. But for each tiling of a $2 \times (n-1)$ rectangle the corresponding $2 \times n$ tiling will have one more vertical tile. For the $2 \times (n-2)$ rectangles no new vertical tiles are added. It follows that the recurrence for these polynomials is
\[p_n(w) = wp_{n-1}(w) + p_{n-2}(w).\]
This is the essence of weighted enumeration.

We have observed above that $p_n(1)$ is the total number of tilings. Observe that $p_n'(1)$, the formal derivative of $p_n(w)$ evaluated at $1$, is the total number of \textbf{all} vertical tiles in all the $2\times n$ tilings. To see this we need only observe that $w^k$ represents a tiling with $k$ vertical dominoes. When you take the derivative you get $kw^{k-1}$, which evaluated at $w = 1$ gives you the number of vertical dominoes in that tiling.

It follows that the average or expected number of vertical tiles in a $2 \times n$ tiling is $\frac{p_n'(1)}{p_n(1)}$. We already know a formula for $p_n(1)$ (as a Fibonacci number). What about the numerator? Here is the magic trick.

Let 
\begin{align*}
P(w, x) &= \sum_{n = 0}^{\infty}p_n(w)x^n \\
&= 1 + wx + (w^2 + 1)x^2 + (w^3 + 2w)x^3 + \cdots
\end{align*}
as a formal power series in $\mathbb{Q}[w][[x]]$. Using the recurrence for the weighted polynomials we can calculate that
\[P(w, x)(q - wx - x^2) = 1 \implies P(w, x) = \frac{1}{1-wx - x^2}\] as the generating function for $P(w, x)$. Observe now (by differentiation with respect to $w$ in the formal power series) that
\begin{align*}
\sum_{n = 0}^{\infty}p_n'(w)x^n &= \frac{d}{dw}\sum_{n = 0}^{\infty}p_n(w)x^n \\
&= \frac{d}{dw}P(w, x) \\
&= \frac{x}{(1 - wx - x^2)^2}.
\end{align*}

So we get a formula for $p_n'(1)$. Letting $\alpha$ and $\beta$ be the reciprocal roots for $1 - x - x^2$ we get a formula of the form $A\alpha^n + B\beta^n + A'n\alpha^n + B'n\beta^n$ for $p_n'(1)$. Letting $\alpha$ be the larger root (in magnitude) this means $p_n'(1) \sim Cn\alpha^n$ for some constant $C$. It follows that approximately,
\[\frac{p_n'(1)}{p_n(1)} = \frac{C_1n\alpha^n}{C_2\alpha^n} = Cn\] for some constant $C$.

Now we consider the general problem of tiling the $2 \times n$ grid where each horizontal domino is given weight $xt$ and each vertical domino is given weight $yt$, and the weight of any given tiling is the product of the tile weights. Then what is the sum of all the weights of all the tilings?

For example, when $n = 1$, the sum of the weights is just $yt$, when $n = 2$ it is $(x^2 + y^2)t^2$ and when $n = 3$ it is $(x^2y + x^2y + y^3)t^3$. Now instead of grouping the sum of tiles by the exponent of $t$ we will instead use the number of blocks, where a block consists of either a vertical domino or two horizontal dominoes. By the multiplication principle for generating functions the generating function looks like

\[\sum_{n = 0}^{\infty}(x^2t^2 + yt)^n = \frac{1}{1 - x^2t^2 - yt}.\]
For example, if $x = 1$ and $y = 1$ then the generating function is as expected (the fibonacci numbers).