We will introduce the Catalan numbers in terms of one of the simplest formulations of object that they count. Let's say that a sequence of $n$ $1$'s and $-1$'s is \textbf{suitable} if every partial sum is greater than or equal to $0$. Call such a sequence
\begin{itemize}
	\item Trivial, if $n = 0$,
	\item Primitive if the partial sums are greater than $0$ except for the empty and total sums,
	\item Composite if some partial sum is $0$.
\end{itemize}
Let $s(n)$, $p(n)$, and $c(n)$ denote these sequences respectively, and let $S(x)$, $P(x)$, and $C(x)$ (bad notation, I know) denote the generating functions for these sequences. We have that 
\[s(n) = p(n) + c(n)\] for $n \geq 1$. Then it follows that 
\begin{enumerate}
	\item $S(x) = 1 + P(x) + C(x)$ (where the $1$ is added to account for the trivial sequence.)
	\item We have
	\[C(x) = P(x)^2 + P(x)^3 + P(x)^4 + \cdots = P(x)[P(x) + C(x)]\] as the consideration that any composite sequence can be thought about as a concatenation of primitive sequences. The latter equality also follows from the fact that each composite sequence starts with either a prime sequence and then follows with either a prime or composite sequence. Using this latter equation we deduce the equality
\[C(x) = \frac{P(x)^2}{1 - P(x)}.\]
	\item Observe that $P(x) = xS(x)$, since prime sequences are essentially determined by the middle portions. It follows that $p(n) = s(n-1)$ for $n \geq 1$.
\end{enumerate}

From these 3 considerations we conclude that
\begin{align*}
S(x) &= 1 + P(x) + \frac{P(x)^2}{1 - P(x)} \\
&= 1 + \frac{P(x)}{1-  P(x)} \\
&= \frac{1}{1 - xS(x)}.
\end{align*}

Rearranging we get the quadratic equation 
\[xS(x)^2 - S(x) + 1 = 0\] for which if we solve it we get
\[S(x) = \frac{1 - \sqrt{1 - 4x}}{2x}\] (for if we take the $+$ root $S(x)$ is no longer a power series but a Laurent series). So 
\[S(x) = \frac{1 - (1 - 2x - 2x^2 - 4x^3 - \cdots)}{2x}.\]
We can do better and get an explicit formula: (derivation goes here)

and hence we see that $C_n = \frac{1}{n + 1}\binom{2n}{n}$. This combinatorial coefficient here suggests we can also prove this formula directly using some kind of equivalence class.

