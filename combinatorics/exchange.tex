It's well known that given an $m \times n$ grid, the number of paths where one can travel along the basis vectors $(1, 0)$ and $(0, 1)$ is $\binom{m + n}{n}$. This approach is also easily generalized for lattice paths in three dimensions: the number of lattice paths of length $(a, b, c)$ from $(0, 0, 0)$ to $(a, b, c)$ is the multinomial coefficient 
\[\binom{a + b + c}{a, b, c} = \frac{(a + b + c)!}{a!b!c!}.\]

But when we move to three space, we have an extra dimension to work with. So now we might want to consider the number of lattice \textit{surfaces}. Consider the followingn problem: what is the number of lattice surfaces of area $ab + bc + ac$ whose boundary is the non-planar hexagon with vertices
\[(a, 0, 0), (a, b, 0), (0, b, 0), (0, b, c), (0, 0, c), (a, 0, c)\]
As a special case for example, what is the number of Art of Problem Solving Default Avatars? We will prove later that the number of such surfaces is
\[\frac{H(a + b + c)H(a)H(b)H(c)}{H(a + b)H(a + c)H(b + c)}\] where $H(n)$ is the hyperfactorial $1!2!3!\cdots (n-1)!$.
We'll prove this by bijection to a two dimensional picture (much like the AoPS default avatars) and we will show that this is the number of ways of tiling an $(a,b,c,a,b,c)$ hexagon with unit rhombuses. For $c = 0$ the number of tilings is $1$ (since the placement of rhombuses is forced).

When $c = 1$, the number of tilings is $\frac{(a + b)!}{a!b!}$. This looks very familiar! This can be illustrated by putting some dots through the vertical sections of the tilings. If a dot hits a vertical line, then we draw a path through it. In this way, we actually always get a lattice path.

With $2 \times n$ domino tilings, we can place alternating dots in a certain way and we will so that the tilings count lattice paths of the directed acyclic graph (put the graph here).

For the $3 \times n$ domino tilings, we might place our dots like so (here) in which case the tilings correspond to paths from $s$ to $t$.

But alternatively for this tiling, the tilings instead correspond to non-intersecting lattice paths (called $2$-routings). The number of such paths will be $N(s_1, t_1)N(s_2, t_2) - N(s_1, t_2)N(s_2, t_1)$. To prove this we can use the reflection principle to show that bad rountings where paths intersect can be directly bijected with paths which cross over each other.

Here is the exchange principle, formally stated:
\begin{theorem}
Suppose $s_1$ and $s_2$ are sources in a directed acyclic graph and $t_1$ and $t_2$ are sinks with the property that a path that joins $s_1$ to $t_2$ and a path that joins $s_2$ to $t_1$ must have a vertex in common.

Then the number of $2$-routings that join $s_1$ to $t_1$ and $s_2$ to $t_2$ with non-intersecting paths is equal to $N_{11}N_{22} - N_{12}N_{21}$ where $N_{ij}$ is the number of paths from $s_i$ to $t_j$.
\end{theorem}

